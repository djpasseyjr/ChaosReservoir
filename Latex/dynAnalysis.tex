

\documentclass[12pt]{article}
\usepackage[letterpaper,margin=1in]{geometry}
\usepackage{times}
\usepackage{graphicx}
\usepackage{sectsty}
\allsectionsfont{\normalsize}
\usepackage{authblk}
\usepackage{amsmath}
\usepackage{amsfonts}
\renewcommand\Authfont{\normalsize}

\begin{document}

\title{\normalsize\bf%
Reservoir Dynamical System Analysis}
\vspace{-0.5in}
\author{\vspace{-0.25in} J Jamieson, D. Passey, D. Smith,  B. Webb, and J. Wilkes\\
}

\date{\vspace{-5ex}} % to kill the unnecessary blank space for \date{}

\maketitle

\thispagestyle{empty}
\pagestyle{empty}


\normalsize \vspace{-.25in}
\section{General 3D Reservoir System}
The reservoir computer dynamical system is
\[
\mathbf{\dot{r}}(t) = \gamma \big[ -\mathbf{r}(t) + 
\tanh  \big(A\mathbf{r}(t) +  W_\text{in} \mathbf{u}(t)\big) \big]
\]
We will simplify the system to a three dimensional system. Write
\[\mathbf{r}(t) = \begin{bmatrix} x(t) & y(t)  & z(t) \end{bmatrix}^T
\] and
\[ W_\text{in} \mathbf{u}(t) = \begin{bmatrix} a(t) & b(t) & c(t) \end{bmatrix}^T
\]
Then the entire system is:
\[
\begin{bmatrix} \dot{x}(t) \\ \dot{y}(t) \\ \dot{z}(t) \end{bmatrix}
=
\gamma \begin{bmatrix}
-x(t) + \tanh\Big(a_{11} x(t) + a_{12} y(t) + a{13} z(t) + a(t) \Big) \\
-y(t) + \tanh\Big(a_{21} x(t) + a_{22} y(t) + a{23} z(t) + b(t) \Big)  \\
-z(t) + \tanh\Big(a_{31} x(t) + a_{32} y(t) + a{33} z(t) + c(t) \Big)
\end{bmatrix}
\]
 Let's investigate a few different adjacency matrices:
 \subsection{Single Path}
 Let
 \[
 A = 
 \begin{bmatrix} 
 0 & 0 & 0 \\
 1 & 0 & 0 \\
 0 & 1 & 0 \\
 \end{bmatrix}
 \]
 
 Then our system simplifies to:
 \[
\begin{bmatrix} \dot{x}(t) \\ \dot{y}(t) \\ \dot{z}(t) \end{bmatrix}
=
\gamma \begin{bmatrix}
-x(t) + \tanh\Big( a(t) \Big) \\
-y(t) + \tanh\Big( x(t) + b(t) \Big)  \\
-z(t) + \tanh\Big( y(t) + c(t) \Big)
\end{bmatrix}
\]

We solve for the location of the fixed point:
 \[
\begin{bmatrix} 0 \\ 0 \\ 0  \end{bmatrix}
=
\gamma \begin{bmatrix}
-x(t) + \tanh\Big( a(t) \Big) \\
-y(t) + \tanh\Big( x(t) + b(t) \Big)  \\
-z(t) + \tanh\Big( y(t) + c(t) \Big)
\end{bmatrix}
\]
\[
\begin{bmatrix} x^\star(t) \\ y^\star(t) \\ z^\star(t) \end{bmatrix}
=
\begin{bmatrix}
\tanh\Big( a(t) \Big) \\
\tanh\Big( x^\star(t) + b(t) \Big)  \\
\tanh\Big( y^\star(t) + c(t) \Big)
\end{bmatrix}
\]
Then, substitute in the solution for $x^\star$ into the equation for $y^\star$ and the solution for $y^\star$ into $z^\star$ to obtain:
\[
x^\star = \tanh\big( a(t) \big)
\]
\[
y^\star = \tanh\big( \tanh\big( a(t) \big) + b(t) \big)
\]
\[
z^\star = \tanh \big( \tanh\big( \tanh\big( a(t) \big) + b(t) \big) + c(t) \big)
\]

The interpretation of this solution is that the location of the fixed point depends only on the input signals. The first reservoir node state $x(t)$ responds only to the input driving it directly $a(t)$. The second reservoir node state $y(t)$ responds to the input driving it directly, $b(t)$ as well as to $\tanh(a(t))$, the transformed input to $x(t)$. This means that input from $a(t)$ will have an impact on the location of $y^\star$, but it will mapped to the interval $(-1, 1)$. Therefore, if $b(t)$ is a lot larger in magnitude, $a(t)$ be overpowered by $b(t)$.

\subsection{Three Node Loop}

Let
 \[
 A = 
 \begin{bmatrix} 
 0 & 0 & 1 \\
 1 & 0 & 0 \\
 0 & 1 & 0 \\
 \end{bmatrix}
 \]
 
 Then our system simplifies to:
 \[
\begin{bmatrix} \dot{x}(t) \\ \dot{y}(t) \\ \dot{z}(t) \end{bmatrix}
=
\gamma \begin{bmatrix}
-x(t) + \tanh\Big( z(t) + a(t) \Big) \\
-y(t) + \tanh\Big( x(t) + b(t) \Big)  \\
-z(t) + \tanh\Big( y(t) + c(t) \Big)
\end{bmatrix}
\]

We solve for the location of the fixed point:
 \[
\begin{bmatrix} 0 \\ 0 \\ 0  \end{bmatrix}
=
\gamma \begin{bmatrix}
-x(t) + \tanh\Big( z(t) + a(t) \Big) \\
-y(t) + \tanh\Big( x(t) + b(t) \Big)  \\
-z(t) + \tanh\Big( y(t) + c(t) \Big)
\end{bmatrix}
\]
\[
\begin{bmatrix} x^\star \\ y^\star \\ z^\star \end{bmatrix}
=
\begin{bmatrix}
\tanh\Big( z^\star + a(t) \Big) \\
\tanh\Big( x^\star + b(t) \Big)  \\
\tanh\Big( y^\star + c(t) \Big)
\end{bmatrix}
\]
Then, substitute in the solution for $x^\star$ into the equation for $y^\star$ and the solution for $y^\star$ into $z^\star$ to obtain:
\[
x^\star = \tanh\Big( \tanh\Big( \tanh\Big( x^\star + b(t) \Big) + c(t) \Big) + a(t) \Big) 
\]
\[
y^\star = \tanh\big( \tanh\Big(\tanh\Big( y^\star + c(t) \Big) + a(t) \Big)  + b(t) \big)
\]
\[
z^\star = \tanh \big( \tanh\Big( \tanh\Big( z^\star + a(t) \Big) + b(t) \Big)  + c(t) \big)
\]


\begin{thebibliography}{9}
\bibitem{review} 
Tanaka, Gouhei, et. al. "Recent advances in physical reservoir computing: A review",
\textit{Neural Networks}, vol. 115. (2019).

\end{thebibliography}


\end{document}